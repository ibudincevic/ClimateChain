\documentclass[]{article}

%opening
\title{}
\author{}

\begin{document}
\title{ClimateChain}
\maketitle

\begin{abstract}
A decentralized project aiming to provide freely available data regarding climate measurements, human energy collection and consumption, while also introducing an economical incentive in making it so. Having such data available to everyone on the planet free of cost, would aid humanity in the fight to protect its survival, by stimulating the growth and creation of many projects aiming to reduce the effects of climate change. 
\end{abstract}

\section{Introduction}
It is not the responsibility of this project to defend the evidence of climate change as a direct result of human activity. The research speaks for itself and it is readily available for anyone to investigate. The project also does not claim that data regarding climate measurements, human energy collection and consumption is not already available and that there are great efforts into improving this availability from many institutions. The main added value that this project hopes to achieve, is using the distributed ledger technology in providing an economical incentive for making this data more reliable and available. 


The motivation behind this approach is based on the preposition that human motivation, to a certain degree, is driven by economic incentives. While there are many goals that people find important in life, the necessity to provide basic needs and a certain standard of life, causes the economic factor to play a significant role in the activities people perform. While distributed ledger technology can provide many advantages towards solving a problem, the crucial one in this line of thought is the aspect of tokenization and thereby rewarding participants of the system. If providing more information about local meteorological metrics such as temperature, pressure, air moisture etc, could be something done by every person owning a simple, cheap IoT (internet of things) device, then if everyone doing so would also get rewarded for this, many private individuals and corporations would be incentivised  to do so. As the network of participants in this system grows, so would the completeness, statistical reliability and granularity of this data. The participants benefit from rewards in terms of ClimateChain tokens, which would be exchangeable for value as any other cryptocurrency, while the mentioned data benefits would stimulate research, efforts into improving energy collection and consumption, but also hopefully stimulating the sharing of energy between individuals.

\section{Token economics}

\section{Technology stack}
\subsection{Distributed ledger technology}
The term "distributed ledger technology" is used purposefully in this text, since it is the authors belief that the question of whether blockchain technology or something like the Tangle of the IOTA project, would be the most suitable for attaining the goals outlined here.  
\subsection{IoT}
\subsection{Integration with existing infrastructure}


\section{Work on the project}
\subsection{Governance}
\section{Roadmap}

\section{Securing funding}

\section{Closing thoughts}
The idea for this project came about from the ideas presented in The Third Industrial revolution and the general available potential of distributed ledger technology. If there is already a project aiming to achieve the same goals which are outlined here, then I would ask anyone aware of this to point me towards it so that I may direct my efforts in helping that project. If this is however not the case at this time, then I would hope this idea would stimulate others who believe this is something worth pursuing to contribute by any means they can to getting this project of the ground. If I realize from the general sentiment that this is not something worth pursuing, I will take it as a sign to direct my efforts in another direction.

Thank you all for your time

 

\end{document}
